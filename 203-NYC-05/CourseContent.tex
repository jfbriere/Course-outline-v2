\entry{Course content}{
The material to be covered is contained in the following chapters and sections of {\bf Physics for Scientists and Engineers by Serway \& Jewett, 9th edition}.
\begin{center}
  \begin{tabulary}{0.95\textwidth}{|c|L|L|}
    \hline
    Weeks & Topics & Chapter \& Section \\ \hline\hline
   1--2 & Periodic motion & Ch.15: 1--5 (physical pendulum optional); sections 6 and 7 qualitatively (quantitative optional for 6, 7)\\ \hline
	2--4 & Mechanical waves & Ch.16: 1--5 (6 optional)\\ \hline
	4--6 & Sound waves and hearing & Ch.17: 1 and 2 qualitatively; 3, 4\\ \hline
	6--7 & Superposition and standing waves & Ch.18: 1--5, 7; 8 qualitatively\\ \hline
	8 & Electromagnetic waves & Ch.34: 7 (EM spectrum)\\ \hline
	8 & Nature and propagation of light & Ch.35: all; Ch.38: 6 qualitatively\\ \hline
	9 & Interference & Ch.37: 1--5\\ \hline
	10 & Diffraction & Ch.38: 1, 2 (intensity optional), 3, 4 without derivations\\ \hline
	11	& Relativity	& Ch.39: 1, 3, 4, 7, 8 (2, 5, 6 optional)\\ \hline
	12--13 & Introduction to quantum mechanics & Ch.40: 1, 2, 4, 5, 7 (8 optional)\\ \hline
	14 & Atomic physics & Ch.42: 1--3 (9, 10 optional)\\ \hline
	14--15 & Nuclear physics & Ch.44: 1, 2, 4--6 (8 optional)\\ \hline
	15 & Applications of nuclear physics & Ch.45: (all optional)\\ \hline
  \end{tabulary}
\end{center}
}
%
%\entry{\mbox{ }}{

%The material to be covered is contained in the following chapters and sections of {\bf University Physics by Young \& Freedman, 13th edition}.
%\begin{center}
%  \begin{tabulary}{0.95\textwidth}{|c|L|L|}
%    \hline
%    Weeks & Topics & Chapter \& Section \\ \hline\hline
%   1--2 & Periodic motion & Ch.14: 1--5; 7, 8 qualitatively\\ \hline
%	3--5 & Mechanical waves & Ch.15: all (derivation of wave equation optional)\\ \hline
%	5--7 & Sound and hearing & Ch.16: 1 qualitatively, 3 (Decibel scale only), 4--6, 8  (Doppler effect for EM waves and 9 optional)\\ \hline
%	8 & Electromagnetic waves & Ch.32: 1 qualitatively\\ \hline
%	8 & Nature and propagation of light & Ch.33: 1--4 (5--7 optional)\\ \hline
%	9 & Interference & Ch.35: 1--4 (5 optional)\\ \hline
%	10 & Diffraction & Ch.36: 1, 2, 5, 7 (3, 4 optional)\\ \hline
%	11	& Relativity	& Ch.37: (all optional)\\ \hline
%	12 & Photons: Light waves behaving as particles & Ch.38: 1, 4 (8 optional)\\ \hline
%	13 & Particles behaving as waves & Ch.39: 1--3, 5 (6 optional)\\ \hline
%	14 & Quantum mechanics & Ch.40: (all optional)\\ \hline
%	14--15 & Nuclear physics & Ch.43: 1 (spins optional), 2 (excluding models), 3, 4, (5--8 optional)\\ \hline
%  \end{tabulary}
%\end{center}
%}
%
\entry{\mbox{ }}{The material to be covered is contained in the following chapters and sections of {\bf Physics for Scientists and Engineers by Knight, 4th edition}.
\begin{center}
  \begin{tabulary}{0.95\textwidth}{|c|L|L|}
    \hline
    Weeks & Topics & Chapter \& Section \\ \hline\hline
   1--2 & Oscillations & Ch.15: 1--6 (physical pendulum optional), 7--8 qualitatively (quantitative optional for 6, 7)\\ \hline
	3--4 & Travelling waves & Ch.16: 1--3, 5, 6 (qualitatively), 7--9 (4 optional)\\ \hline
	5--6 & Superposition & Ch.17: 1--8\\ \hline
	7--8 & Wave optics & Ch.33: 1--7\\ \hline
	9 & Ray optics, Rayleigh's Criterion & Ch.34: 1--3 Ch.35: 5--6\\ \hline
	10	& Relativity	& Ch.36: 3, 6, 7, 9 and 10 (1, 2, 4, 5, 8 optional)\\ \hline
	11 & Foundations of modern physics & Ch.37: 1, 2 (3--8 qualitatively)\\ \hline
	11--13	& Quantization & Ch.38: 1--7\\ \hline
	14 & Wave functions and uncertainty & Ch.39: (6 optional) \\ \hline
	14--15 & Nuclear physics & Ch.42: 1--3, 5, 6 (4, 7 optional)\\ \hline
  \end{tabulary}
\end{center}
}
